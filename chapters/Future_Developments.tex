\chapter{Future Developments}
More advanced Information Retrieval techniques can be used to improve the core matcher performances. 
Learn to rank, or more sophisticated query expansions ways that could involve the recent progress made in the field of AI, and specifically Language Models. 
\section{Software Identification remains an unsolved problem}
This is indeed a problem that lies in the way language is used to describe the same thing in different ways
Requirements for an agreed naming method should allow anyone to obtain the same name for the same element, and conversely to have only a single to for each element. 
\subsection{Software Identification Ecosystem Option Analysis by CISA}
\subsection{Intrinsic and Extrinsic tags}
((TODO))
\subsubsection{Package URL}
((TODO))
\subsubsection{Software Identification Tags}

Software Identification (SWID) Tags, defined by the ISO/IEC 19770-2:2015 standard, promise to be an important step towards helping organizations to identify. SWID Tags provide a transparent way for organizations to track the software installed on their managed devices. SWID Tag files contain descriptive information about a specific release of a software product. The SWID standard defines a lifecycle where a SWID Tag is added to an endpoint as part of the software product’s installation process and deleted by the product’s uninstall process. When this lifecycle is followed, the presence of a given SWID Tag corresponds directly to the presence of the software product that the Tag describes. The National Institute of Standards and Technology recommends adoption of the SWID Tag standard by software producers, and multiple standards bodies, including the Trusted Computing Group (TCG) and the Internet Engineering Task Force (IETF) utilize SWID Tags in their standards.

NIST plans to continue to promote the incorporation of the SWID Tag standard and associated guidelines in other international consensus standards (such as IETF and TCG efforts), the broad adoption of SWID tagging within the software community, and the use of SWID Tag information in the creation of cybersecurity reference data and security automation content. Additionally, NIST is working to incorporate SWID Tag data into the vulnerability dataset provided by National Vulnerability Database (NVD), and has incorporated use of SWID Tag data into Security Content Automation Protocol (SCAP) version 1.3. These efforts are part of the Software Identification (SWID) Tagging project, which is an initiative of the Computer Security Division’s Security Automation Program (SAP). The SAP is focused on standardizing the exchange of security posture information supporting the management of software, vulnerabilities, patches, and secure configurations for computing devices.
\section{Alternatives approaches}
In the meantime, there can be alternative approaches to just CVE filtering.
For example, machine learning techniques to inspect new vulnerability reports and somehow find the relevant vulnerably software in the inventory

Generative architectures do not seem to be appropriate for this task, unless it is guaranteed that the same identification is always generated for the same software
Efforts like secureBERT might be useful to recognie relevant segments inside a vulnerability report, which would help identifying the affected software,